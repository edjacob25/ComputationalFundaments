\documentclass{article}
    % General document formatting
    \usepackage[margin=0.7in]{geometry}
    \usepackage[parfill]{parskip}
    \usepackage[utf8]{inputenc}
    \usepackage{amsmath}
    \usepackage{amssymb}
    \usepackage{tikz}
    \usepackage{fancyhdr}
    \usepackage{listings}
    \usepackage{multicol}
    \usepackage{polynom}

\pagestyle{fancy}
\fancyhf{}
\rhead{Edgar Jacob Rivera Rios - A01184125}

\begin{document}
\begin{titlepage}

    \newcommand{\HRule}{\rule{\linewidth}{0.5mm}} % Defines a new command for the horizontal lines, change thickness here

    \center % Center everything on the page

    %----------------------------------------------------------------------------------------
    %	HEADING SECTIONS
    %----------------------------------------------------------------------------------------

    \textsc{\LARGE Tecnológico de Monterrey}\\[1.5cm] % Name of your university/college
    \textsc{\Large Fundamentos de computación}\\[0.5cm] % Major heading such as course name
    %\textsc{\large Minor Heading}\\[0.5cm] % Minor heading such as course title

    %----------------------------------------------------------------------------------------
    %	TITLE SECTION
    %----------------------------------------------------------------------------------------

    \HRule \\[0.4cm]
    { \huge \bfseries Homework 8}\\[0.4cm] % Title of your document
    \HRule \\[1.5cm]

    %----------------------------------------------------------------------------------------
    %	AUTHOR SECTION
    %----------------------------------------------------------------------------------------

    \begin{minipage}{0.4\textwidth}
    \begin{flushleft} \large
    \emph{Student:}\\
    Jacob \textsc{Rivera} % Your name
    \end{flushleft}
    \end{minipage}
    ~
    \begin{minipage}{0.4\textwidth}
    \begin{flushright} \large
    \emph{Professor:} \\
    Dr. Hugo \textsc{Terashima} % Supervisor's Name
    \end{flushright}
    \end{minipage}\\[2cm]

    % If you don't want a supervisor, uncomment the two lines below and remove the section above
    %\Large \emph{Author:}\\
    %John \textsc{Smith}\\[3cm] % Your name

    %----------------------------------------------------------------------------------------
    %	DATE SECTION
    %----------------------------------------------------------------------------------------

    {\large \today}\\[2cm] % Date, change the \today to a set date if you want to be precise

    %----------------------------------------------------------------------------------------
    %	LOGO SECTION
    %----------------------------------------------------------------------------------------

    \includegraphics[width=0.4\textwidth,height=\textheight,keepaspectratio]{logo-tec-negro.png} % Include a department/university logo - this will require the graphicx package

    %----------------------------------------------------------------------------------------

    \vfill % Fill the rest of the page with whitespace

\end{titlepage}


\section{Problems}
Solve the following problems:
\begin{enumerate}
    \item Investigate the algorithm for computing the maximum flow on a graph. Provide and example and apply the algorithm, showing each step.
    \item Given a graph $G = (V, E, W)$ and a MST $T$, suppose that we decrease the weight of one of the edges not in $T$. Design an algorithm and its computational complexity to find the MST in the modified graph.
    \begin{figure}[h]
        \centering
        \includegraphics{Homework8.png}
        \caption{Graph}
        \label{graph}
    \end{figure}
    \item Obtain the MST for the graph in Figure 1 using both the Kruskal and Dijastra/Prim algorithms.
    \item  Design a graph with at least 4 components (biconnected, each with three or more nodes) and run the algorithm seen in class to obtain them. Show the steps.
    \item Design a directed graph and establish an origin and a destination and apply the algorithm Dijkstra/Prim to obtain the shortest path between both nodes. Show the steps.
    
\end{enumerate}
\end{document}