\documentclass{article}
    % General document formatting
    \usepackage[margin=0.7in]{geometry}
    \usepackage[parfill]{parskip}
    \usepackage[utf8]{inputenc}
    \usepackage{amsmath}
    \usepackage{amssymb}
    \usepackage{tikz}
    \usepackage{fancyhdr}
    \usepackage{listings}
    \usepackage{multicol}
    \usepackage{polynom}

\pagestyle{fancy}
\fancyhf{}
\rhead{Edgar Jacob Rivera Rios - A01184125}

\begin{document}
\begin{titlepage}

    \newcommand{\HRule}{\rule{\linewidth}{0.5mm}} % Defines a new command for the horizontal lines, change thickness here

    \center % Center everything on the page

    %----------------------------------------------------------------------------------------
    %	HEADING SECTIONS
    %----------------------------------------------------------------------------------------

    \textsc{\LARGE Tecnológico de Monterrey}\\[1.5cm] % Name of your university/college
    \textsc{\Large Fundamentos de computación}\\[0.5cm] % Major heading such as course name
    %\textsc{\large Minor Heading}\\[0.5cm] % Minor heading such as course title

    %----------------------------------------------------------------------------------------
    %	TITLE SECTION
    %----------------------------------------------------------------------------------------

    \HRule \\[0.4cm]
    { \huge \bfseries Homework 11}\\[0.4cm] % Title of your document
    \HRule \\[1.5cm]

    %----------------------------------------------------------------------------------------
    %	AUTHOR SECTION
    %----------------------------------------------------------------------------------------

    \begin{minipage}{0.4\textwidth}
    \begin{flushleft} \large
    \emph{Student:}\\
    Jacob \textsc{Rivera} % Your name
    \end{flushleft}
    \end{minipage}
    ~
    \begin{minipage}{0.4\textwidth}
    \begin{flushright} \large
    \emph{Professor:} \\
    Dr. Hugo \textsc{Terashima} % Supervisor's Name
    \end{flushright}
    \end{minipage}\\[2cm]

    % If you don't want a supervisor, uncomment the two lines below and remove the section above
    %\Large \emph{Author:}\\
    %John \textsc{Smith}\\[3cm] % Your name

    %----------------------------------------------------------------------------------------
    %	DATE SECTION
    %----------------------------------------------------------------------------------------

    {\large \today}\\[2cm] % Date, change the \today to a set date if you want to be precise

    %----------------------------------------------------------------------------------------
    %	LOGO SECTION
    %----------------------------------------------------------------------------------------

    \includegraphics[width=0.4\textwidth,height=\textheight,keepaspectratio]{logo-tec-negro.png} % Include a department/university logo - this will require the graphicx package

    %----------------------------------------------------------------------------------------

    \vfill % Fill the rest of the page with whitespace

\end{titlepage}


\section{Problems}
Solve the following problems:
\begin{enumerate}
    \item Provide a parallel algorithm for merging two lists of $n/2$ keys each. State the number of processors used and compute the metrics $S_p$, $E_p$ and $R_p$.
    \item Given the binary fan-in technique described in class to calculate the maximum of $n$ numbers, calculate its speed-up ratio and its efficiency with respect to the sequential tournament version of the algorithm.
    \item Prove (and provide an example) that the multiplication of two $n \times n$ matrices can be conducted by a PRAM program in $O(\text{log}_2 n)$ steps if $n^3$ processors are available.
    \item Let a binary operation take $k$ cycles to complete when done serially. If this operation is pipelined using a $k$-segment pipe, show that the resulting speed-up in computing $n$ operations is
    \begin{equation*}
        S_k = \frac{nk}{n + k - 1}
    \end{equation*}
    
\end{enumerate}
\end{document}