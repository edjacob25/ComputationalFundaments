\documentclass{article}
    % General document formatting
    \usepackage[margin=0.7in]{geometry}
    \usepackage[parfill]{parskip}
    \usepackage[utf8]{inputenc}
    \usepackage{amsmath}
    \usepackage{amssymb}
    \usepackage{tikz}
    \usepackage{fancyhdr}
    \usepackage{listings}

\pagestyle{fancy}
\fancyhf{}
\rhead{Edgar Jacob Rivera Rios - A01184125}

\begin{document}
\begin{titlepage}

    \newcommand{\HRule}{\rule{\linewidth}{0.5mm}} % Defines a new command for the horizontal lines, change thickness here

    \center % Center everything on the page

    %----------------------------------------------------------------------------------------
    %	HEADING SECTIONS
    %----------------------------------------------------------------------------------------

    \textsc{\LARGE Tecnológico de Monterrey}\\[1.5cm] % Name of your university/college
    \textsc{\Large Fundamentos de computación}\\[0.5cm] % Major heading such as course name
    %\textsc{\large Minor Heading}\\[0.5cm] % Minor heading such as course title

    %----------------------------------------------------------------------------------------
    %	TITLE SECTION
    %----------------------------------------------------------------------------------------

    \HRule \\[0.4cm]
    { \huge \bfseries Homework 3}\\[0.4cm] % Title of your document
    \HRule \\[1.5cm]

    %----------------------------------------------------------------------------------------
    %	AUTHOR SECTION
    %----------------------------------------------------------------------------------------

    \begin{minipage}{0.4\textwidth}
    \begin{flushleft} \large
    \emph{Student:}\\
    Jacob \textsc{Rivera} % Your name
    \end{flushleft}
    \end{minipage}
    ~
    \begin{minipage}{0.4\textwidth}
    \begin{flushright} \large
    \emph{Professor:} \\
    Dr. Hugo \textsc{Terashima} % Supervisor's Name
    \end{flushright}
    \end{minipage}\\[2cm]

    % If you don't want a supervisor, uncomment the two lines below and remove the section above
    %\Large \emph{Author:}\\
    %John \textsc{Smith}\\[3cm] % Your name

    %----------------------------------------------------------------------------------------
    %	DATE SECTION
    %----------------------------------------------------------------------------------------

    {\large \today}\\[2cm] % Date, change the \today to a set date if you want to be precise

    %----------------------------------------------------------------------------------------
    %	LOGO SECTION
    %----------------------------------------------------------------------------------------

    \includegraphics[width=0.4\textwidth,height=\textheight,keepaspectratio]{logo-tec-negro.png} % Include a department/university logo - this will require the graphicx package

    %----------------------------------------------------------------------------------------

    \vfill % Fill the rest of the page with whitespace

\end{titlepage}


\section{Problems}
Solve the following problems:
\begin{enumerate}
    \item  Find an optimal parenthesization of a matrix-chain product whose sequence of dimensions are $p_0= 10$, $p_1= 19$, $p_2= 29$, $p_3= 33$, $p_4= 9$ and $p_5= 17$. Show and explain each step in the procedure.

    \begin{align*}
        A_1 &= 10\times19\\
        A_2 &= 19\times29\\
        A_3 &= 29\times33\\
        A_4 &= 33\times9\\
        A_5 &= 9\times17\\
    \end{align*}
    \begin{center}
        \begin{tabular}{c|c|c|c|c|c|}
             & 1 & 2 & 3 & 4 & 5 \\
            \hline
            1& 0 & & & & \\
            \hline
            2& 5510 & 0 & & & \\
            \hline
            3& 15080 & 18183 & 0 & & \\
            \hline
            4& 15282 & 13572 & 8613 & 0 & \\
            \hline
            5& 16812 & 16479 & 13050 & 5049 & 0 \\
            \hline
        \end{tabular}

        \begin{tabular}{c|c|c|c|c|}
             & 1 & 2 & 3 & 4\\
            \hline
            2& 1 & & &  \\
            \hline
            3& 2 & 2 & &  \\
            \hline
            4& 1 & 2 & 3 &  \\
            \hline
            5& 4 & 4 & 4 & 4 \\
            \hline
        \end{tabular}
    \end{center}

    \item  Show that a full parenthesization of an $n$ element expression has exactly $n-1$ pairs of parentheses

    We can see that a parenthesization of one element is none and of two elements is trivial, as the only possibility is $(A_1A_2)$, but when we add a third element, we have the possibilities of $((A_1A_2)A_3)$ and $(A_1(A_2A_3))$, so, by in induction we can see that the number of pairs parentheses is dependent of the number of terms, as $n -1$
    \pagebreak
    \item Solve problem 15-4 from Cormen et al. Book.

    Consider the problem of neatly printing a paragraph with a monospaced font (all characters having the same width) on a printer. The input text is a sequence of $n$ words of lengths $l_1, l_2, ... , l_n$, measured in characters. We want to print this paragraph neatly on a number of lines that hold a maximum of $M$ characters each. Our criterion of “neatness” is as follows. If a given line contains words $i$ through $j$, where $i \leq j$, and we leave exactly one space between words, the number of extra space characters at the end of the line is $M - j + i - \sum_{k=i}^{j}l_k$, which must be non negative so that the words fit on the line. We wish to minimize the sum, over all lines except the last, of the cubes of the numbers of extra space characters at the ends of lines. Give a dynamic-programming algorithm to print a paragraph of $n$ words neatly on a printer. Analyze the running time and space requirements of your algorithm.
    \begin{lstlisting}
    PRINT-NEATLY(l[], n, M)
        spaces_at_the_end = new [n][n]
        line_cost = new [n][n]
        c = new [n]
        for i = 1 to n
            spaces_at_the_end[i][i] = M - l[i]
            for j = i + 1 to n
                spaces_at_the_end[i, j] = spaces_at_the_end[i][j - 1] - l[j] - 1
        for i = 1 to n
            for j = i to n
                if spaces_at_the_end[i][j] < 0
                    line_cost[i][j] = inf
                else if j == n and spaces_at_the_end[i][j] >= 0
                    line_cost[i][j] = 0
                else line_cost[i][j] = spaces_at_the_end[i][j]^3
        c[0] = 0
        p = new [n]
        for j = 1 to n
            c[j] = inf
            for i = 1 to j
                if c[i - 1] + line_cost[i][j] < c[j]
                    c[j] = c[i - 1] + line_cost[i][j]
                    p[j] = i
        return c and p
    \end{lstlisting}

    We can see that the complexity is simply $O(n^2)$

    \item Provide a comparative study on an investigation over algorithm-design strategies: Divide and Conquer, Dynamic Programming and Greedy Algorithms. State their definition, characteristics, advantages, disadvantages, examples of problems where each strategy is best applied, and a description to characterize problems within each technique. Add references to your work.

\end{enumerate}
\end{document}